\documentclass[]{article}
\usepackage{lmodern}
\usepackage{amssymb,amsmath}
\usepackage{ifxetex,ifluatex}
\usepackage{fixltx2e} % provides \textsubscript
\ifnum 0\ifxetex 1\fi\ifluatex 1\fi=0 % if pdftex
  \usepackage[T1]{fontenc}
  \usepackage[utf8]{inputenc}
\else % if luatex or xelatex
  \ifxetex
    \usepackage{mathspec}
  \else
    \usepackage{fontspec}
  \fi
  \defaultfontfeatures{Ligatures=TeX,Scale=MatchLowercase}
  \newcommand{\euro}{€}
\fi
% use upquote if available, for straight quotes in verbatim environments
\IfFileExists{upquote.sty}{\usepackage{upquote}}{}
% use microtype if available
\IfFileExists{microtype.sty}{%
\usepackage{microtype}
\UseMicrotypeSet[protrusion]{basicmath} % disable protrusion for tt fonts
}{}


\usepackage{longtable,booktabs}
\usepackage{graphicx,grffile}
\makeatletter
\def\maxwidth{\ifdim\Gin@nat@width>\linewidth\linewidth\else\Gin@nat@width\fi}
\def\maxheight{\ifdim\Gin@nat@height>\textheight\textheight\else\Gin@nat@height\fi}
\makeatother
% Scale images if necessary, so that they will not overflow the page
% margins by default, and it is still possible to overwrite the defaults
% using explicit options in \includegraphics[width, height, ...]{}
\setkeys{Gin}{width=\maxwidth,height=\maxheight,keepaspectratio}
\setlength{\parindent}{0pt}
\setlength{\parskip}{6pt plus 2pt minus 1pt}
\setlength{\emergencystretch}{3em}  % prevent overfull lines
\providecommand{\tightlist}{%
  \setlength{\itemsep}{0pt}\setlength{\parskip}{0pt}}
\setcounter{secnumdepth}{0}

%%% Use protect on footnotes to avoid problems with footnotes in titles
\let\rmarkdownfootnote\footnote%
\def\footnote{\protect\rmarkdownfootnote}

%%% Change title format to be more compact
\usepackage{titling}

\RequirePackage[]{C:/Users/s057g121/Documents/R/win-library/3.5/BiocStyle/resources/tex/Bioconductor}

% Create subtitle command for use in maketitle
\newcommand{\subtitle}[1]{
  \posttitle{
    \begin{center}\large#1\end{center}
    }
}

\setlength{\droptitle}{-2em}

\bioctitle[]{The pwrEWAS User's Guide}
    \pretitle{\vspace{\droptitle}\centering\huge}
  \posttitle{\par}
\author{Stefan Graw, Devin C. Koestler}
    \preauthor{\centering\large\emph}
  \postauthor{\par}
      \predate{\centering\large\emph}
  \postdate{\par}
    \date{13 February 2019}


% Redefines (sub)paragraphs to behave more like sections
\ifx\paragraph\undefined\else
\let\oldparagraph\paragraph
\renewcommand{\paragraph}[1]{\oldparagraph{#1}\mbox{}}
\fi
\ifx\subparagraph\undefined\else
\let\oldsubparagraph\subparagraph
\renewcommand{\subparagraph}[1]{\oldsubparagraph{#1}\mbox{}}
\fi

% code highlighting
\definecolor{fgcolor}{rgb}{0.251, 0.251, 0.251}
\newcommand{\hlnum}[1]{\textcolor[rgb]{0.816,0.125,0.439}{#1}}%
\newcommand{\hlstr}[1]{\textcolor[rgb]{0.251,0.627,0.251}{#1}}%
\newcommand{\hlcom}[1]{\textcolor[rgb]{0.502,0.502,0.502}{\textit{#1}}}%
\newcommand{\hlopt}[1]{\textcolor[rgb]{0,0,0}{#1}}%
\newcommand{\hlstd}[1]{\textcolor[rgb]{0.251,0.251,0.251}{#1}}%
\newcommand{\hlkwa}[1]{\textcolor[rgb]{0.125,0.125,0.941}{#1}}%
\newcommand{\hlkwb}[1]{\textcolor[rgb]{0,0,0}{#1}}%
\newcommand{\hlkwc}[1]{\textcolor[rgb]{0.251,0.251,0.251}{#1}}%
\newcommand{\hlkwd}[1]{\textcolor[rgb]{0.878,0.439,0.125}{#1}}%
\let\hlipl\hlkwb
%
\usepackage{fancyvrb}
\newcommand{\VerbBar}{|}
\newcommand{\VERB}{\Verb[commandchars=\\\{\}]}
\DefineVerbatimEnvironment{Highlighting}{Verbatim}{commandchars=\\\{\}}
%
\newenvironment{Shaded}{\begin{myshaded}}{\end{myshaded}}
% set background for result chunks
\let\oldverbatim\verbatim
\renewenvironment{verbatim}{\color{codecolor}\begin{myshaded}\begin{oldverbatim}}{\end{oldverbatim}\end{myshaded}}
%
\newcommand{\KeywordTok}[1]{\hlkwd{#1}}
\newcommand{\DataTypeTok}[1]{\hlkwc{#1}}
\newcommand{\DecValTok}[1]{\hlnum{#1}}
\newcommand{\BaseNTok}[1]{\hlnum{#1}}
\newcommand{\FloatTok}[1]{\hlnum{#1}}
\newcommand{\ConstantTok}[1]{\hlnum{#1}}
\newcommand{\CharTok}[1]{\hlstr{#1}}
\newcommand{\SpecialCharTok}[1]{\hlstr{#1}}
\newcommand{\StringTok}[1]{\hlstr{#1}}
\newcommand{\VerbatimStringTok}[1]{\hlstr{#1}}
\newcommand{\SpecialStringTok}[1]{\hlstr{#1}}
\newcommand{\ImportTok}[1]{{#1}}
\newcommand{\CommentTok}[1]{\hlcom{#1}}
\newcommand{\DocumentationTok}[1]{\hlcom{#1}}
\newcommand{\AnnotationTok}[1]{\hlcom{#1}}
\newcommand{\CommentVarTok}[1]{\hlcom{#1}}
\newcommand{\OtherTok}[1]{{#1}}
\newcommand{\FunctionTok}[1]{\hlstd{#1}}
\newcommand{\VariableTok}[1]{\hlstd{#1}}
\newcommand{\ControlFlowTok}[1]{\hlkwd{#1}}
\newcommand{\OperatorTok}[1]{\hlopt{#1}}
\newcommand{\BuiltInTok}[1]{{#1}}
\newcommand{\ExtensionTok}[1]{{#1}}
\newcommand{\PreprocessorTok}[1]{\textit{#1}}
\newcommand{\AttributeTok}[1]{{#1}}
\newcommand{\RegionMarkerTok}[1]{{#1}}
\newcommand{\InformationTok}[1]{\textcolor{messagecolor}{#1}}
\newcommand{\WarningTok}[1]{\textcolor{warningcolor}{#1}}
\newcommand{\AlertTok}[1]{\textcolor{errorcolor}{#1}}
\newcommand{\ErrorTok}[1]{\textcolor{errorcolor}{#1}}
\newcommand{\NormalTok}[1]{\hlstd{#1}}
%
\AtBeginDocument{\bibliographystyle{C:/Users/s057g121/Documents/R/win-library/3.5/BiocStyle/resources/tex/unsrturl}}

\begin{document}
\maketitle
\begin{abstract}
pwrEWAS is a user-friendly tool to estimate power in EWAS as a function
of sample and effect size for two-group comparisons of DNAm (e.g., case
vs control, exposed vs non-exposed, etc.). Detailed description of
in-/outputs, instructions and an example, as well as interpretations of
the example results are provided in the following vignette.
\end{abstract}

\packageVersion{pwrEWAS 0.99.0}

{
\setcounter{tocdepth}{2}
\tableofcontents
\newpage
}
\section{Introduction}\label{introduction}

When designing an epigenome-wide association study (EWAS) to investigate
the relationship between DNA methylation (DNAm) and some exposure(s) or
phenotype(s), it is critically important to assess the sample size
needed to detect a hypothesized difference with adequate statistical
power. However, the complex and nuanced nature of DNAm data makes direct
assessment of statistical power challenging. To circumvent these
challenges and to address the outstanding need for a user-friendly
interface for EWAS power evaluation, we have developed pwrEWAS. The
current implementation of pwrEWAS accommodates power estimation for
two-group comparisons of DNAm (e.g.~case vs control, exposed vs
non-exposed, etc.), where methylation assessment is carried out using
the Illumina Human Methylation BeadChip technology. Power is calculated
using a semi-parametric simulation-based approach in which DNAm data is
randomly generated from beta-distributions using CpG-specific means and
variances estimated from one of several different existing DNAm data
sets, chosen to cover the most common tissue-types used in EWAS. In
addition to specifying the tissue type to be used for DNAm profiling,
users are required to specify the sample size, number of differentially
methylated CpGs, effect size(s) ( ), target false discovery rate (FDR)
and the number of simulated data sets, and have the option of selecting
from several different statistical methods to perform differential
methylation analyses. pwrEWAS reports the marginal power, marginal type
I error rate, marginal FDR, and false discovery cost (FDC). The R-Shiny
web inter-face allows for easy input of user-defined parameters and
includes an advanced settings button that offers additional options
pertaining to data generation and computation.

\subsection{Installation}\label{installation}

pwrEWAS can be installed from guthub with the following R code:

\begin{Shaded}
\begin{Highlighting}[]
\CommentTok{# devtools::install_github("stefangraw/pwrEWAS")}
\KeywordTok{library}\NormalTok{(pwrEWAS)}
\end{Highlighting}
\end{Shaded}

\newpage

\section{Usage}\label{usage}

To execute the main pwrEWAS function the following to codes can be used.
pwrEWAS allows the user to specify the effect size in one of two ways,
by either providing a target maximal difference in methylation
(``targetDelta''), or by providing the standard deviation of the
simulated differnces (``deltaSD''). Only one of both arguments can be
provided. If ``targetDelta'' is specified, pwrEWAS will automatically
identify a standard deviation to simulate differences in methylation,
such that the 99.99th percentile of the absolute value of simulated
differences falls within a range around the targeted maximal difference
in DNAm (see paper for additional details). If ``deltaSD'' is specified,
pwrEWAS will simulate differences in methylation using the provided
standard deviation (additional information provided in paper).

\begin{Shaded}
\begin{Highlighting}[]
\CommentTok{# providing the targeted maximal difference in DNAm}
\NormalTok{results_targetDelta =}\StringTok{ }\KeywordTok{pwrEWAS}\NormalTok{(}\DataTypeTok{minTotSampleSize =} \DecValTok{10}\NormalTok{,}
                    \DataTypeTok{maxTotSampleSize =} \DecValTok{50}\NormalTok{,}
                    \DataTypeTok{SampleSizeSteps =} \DecValTok{10}\NormalTok{,}
                    \DataTypeTok{NcntPer =} \FloatTok{0.5}\NormalTok{,}
                    \DataTypeTok{targetDelta =} \KeywordTok{c}\NormalTok{(}\FloatTok{0.2}\NormalTok{, }\FloatTok{0.5}\NormalTok{),}
                    \DataTypeTok{J =} \DecValTok{100}\NormalTok{,}
                    \DataTypeTok{targetDmCpGs =} \DecValTok{10}\NormalTok{,}
                    \DataTypeTok{tissueType =} \StringTok{"Adult (PBMC)"}\NormalTok{,}
                    \DataTypeTok{detectionLimit =} \FloatTok{0.01}\NormalTok{,}
                    \DataTypeTok{DMmethod =} \StringTok{"limma"}\NormalTok{,}
                    \DataTypeTok{FDRcritVal =} \FloatTok{0.05}\NormalTok{,}
                    \DataTypeTok{core =} \DecValTok{4}\NormalTok{,}
                    \DataTypeTok{sims =} \DecValTok{50}\NormalTok{)}

\CommentTok{# providing the targeted maximal difference in DNAm}
\NormalTok{results_deltaSD =}\StringTok{ }\KeywordTok{pwrEWAS}\NormalTok{(}\DataTypeTok{minTotSampleSize =} \DecValTok{10}\NormalTok{,}
                    \DataTypeTok{maxTotSampleSize =} \DecValTok{50}\NormalTok{,}
                    \DataTypeTok{SampleSizeSteps =} \DecValTok{10}\NormalTok{,}
                    \DataTypeTok{NcntPer =} \FloatTok{0.5}\NormalTok{,}
                    \DataTypeTok{deltaSD =} \KeywordTok{c}\NormalTok{(}\FloatTok{0.02}\NormalTok{, }\FloatTok{0.05}\NormalTok{),}
                    \DataTypeTok{J =} \DecValTok{100}\NormalTok{,}
                    \DataTypeTok{targetDmCpGs =} \DecValTok{10}\NormalTok{,}
                    \DataTypeTok{tissueType =} \StringTok{"Adult (PBMC)"}\NormalTok{,}
                    \DataTypeTok{detectionLimit =} \FloatTok{0.01}\NormalTok{,}
                    \DataTypeTok{DMmethod =} \StringTok{"limma"}\NormalTok{,}
                    \DataTypeTok{FDRcritVal =} \FloatTok{0.05}\NormalTok{,}
                    \DataTypeTok{core =} \DecValTok{4}\NormalTok{,}
                    \DataTypeTok{sims =} \DecValTok{50}\NormalTok{)}
\end{Highlighting}
\end{Shaded}

\newpage

\subsection{Input parameter}\label{input-parameter}

The following table provides a description of the input arguments:

\begin{longtable}[]{@{}ll@{}}
\toprule
\begin{minipage}[b]{0.18\columnwidth}\raggedright\strut
Parameter\strut
\end{minipage} & \begin{minipage}[b]{0.76\columnwidth}\raggedright\strut
Decription\strut
\end{minipage}\tabularnewline
\midrule
\endhead
\begin{minipage}[t]{0.18\columnwidth}\raggedright\strut
minTotSampleSize\strut
\end{minipage} & \begin{minipage}[t]{0.76\columnwidth}\raggedright\strut
Lowest total sample sizes to be considered\strut
\end{minipage}\tabularnewline
\begin{minipage}[t]{0.18\columnwidth}\raggedright\strut
maxTotSampleSize\strut
\end{minipage} & \begin{minipage}[t]{0.76\columnwidth}\raggedright\strut
Highest total sample sizes to be considered\strut
\end{minipage}\tabularnewline
\begin{minipage}[t]{0.18\columnwidth}\raggedright\strut
SampleSizeSteps\strut
\end{minipage} & \begin{minipage}[t]{0.76\columnwidth}\raggedright\strut
Steps with which total sample size increases from minTotSampleSize to
maxTotSampleSize\strut
\end{minipage}\tabularnewline
\begin{minipage}[t]{0.18\columnwidth}\raggedright\strut
NcntPer\strut
\end{minipage} & \begin{minipage}[t]{0.76\columnwidth}\raggedright\strut
Rate by which the total sample size is split into groups (0.5
corresponds to a balanced study; rate for group 2 is equal to 1 rate of
group 1)\strut
\end{minipage}\tabularnewline
\begin{minipage}[t]{0.18\columnwidth}\raggedright\strut
targetDelta\strut
\end{minipage} & \begin{minipage}[t]{0.76\columnwidth}\raggedright\strut
Standard deviations of the simulated differences is automatically
determined such that the 99\%til of the simulated differences are within
a range around the provided values\strut
\end{minipage}\tabularnewline
\begin{minipage}[t]{0.18\columnwidth}\raggedright\strut
deltaSD\strut
\end{minipage} & \begin{minipage}[t]{0.76\columnwidth}\raggedright\strut
Differences in methylation will be simulated using provided standard
deviation\strut
\end{minipage}\tabularnewline
\begin{minipage}[t]{0.18\columnwidth}\raggedright\strut
J\strut
\end{minipage} & \begin{minipage}[t]{0.76\columnwidth}\raggedright\strut
Number of CpG site that will simulated and tested (increasing Number of
CpGs tested will require increasing RAM (memory))\strut
\end{minipage}\tabularnewline
\begin{minipage}[t]{0.18\columnwidth}\raggedright\strut
targetDmCpGs\strut
\end{minipage} & \begin{minipage}[t]{0.76\columnwidth}\raggedright\strut
Target number of CpGs simulated with meaningful differences (differences
greater than detection limit)\strut
\end{minipage}\tabularnewline
\begin{minipage}[t]{0.18\columnwidth}\raggedright\strut
tissueType\strut
\end{minipage} & \begin{minipage}[t]{0.76\columnwidth}\raggedright\strut
Heterogeneity of different tissue types can have effects on the results.
Please select your tissue type of interest or one you believe is the
closest\strut
\end{minipage}\tabularnewline
\begin{minipage}[t]{0.18\columnwidth}\raggedright\strut
detectionLimit\strut
\end{minipage} & \begin{minipage}[t]{0.76\columnwidth}\raggedright\strut
Limit to detect changes in methylation. Simulated differences below the
detection limit will not be consider as meaningful differentially
methylated CpGs\strut
\end{minipage}\tabularnewline
\begin{minipage}[t]{0.18\columnwidth}\raggedright\strut
DMmethod\strut
\end{minipage} & \begin{minipage}[t]{0.76\columnwidth}\raggedright\strut
Method used to perform differential methylation analysis\strut
\end{minipage}\tabularnewline
\begin{minipage}[t]{0.18\columnwidth}\raggedright\strut
FDRcritVal\strut
\end{minipage} & \begin{minipage}[t]{0.76\columnwidth}\raggedright\strut
Critical value to control the False Discovery Rate (FDR) using the
Benjamini and Hochberg method\strut
\end{minipage}\tabularnewline
\begin{minipage}[t]{0.18\columnwidth}\raggedright\strut
core\strut
\end{minipage} & \begin{minipage}[t]{0.76\columnwidth}\raggedright\strut
Number of cores used to run multiple threads. Ideally, the number of
different total samples sizes multiplied by the number of effect sizes
should be a multiple (m) of the number of cores (\#sampleSizes *
\#effectSizes = m * \#threads). An increasing number of threads will
require an increasing amount of RAM (memory)\strut
\end{minipage}\tabularnewline
\begin{minipage}[t]{0.18\columnwidth}\raggedright\strut
sims\strut
\end{minipage} & \begin{minipage}[t]{0.76\columnwidth}\raggedright\strut
Number of repeated simulation/simulated data sets under the same
conditions for consistent results\strut
\end{minipage}\tabularnewline
\bottomrule
\end{longtable}

\subsection{Output parameter}\label{output-parameter}

Running pwrEWAS will result in an object with the following four
attributes: meanPower, powerArray, deltaArray, and metric. The first
attribute `'meanPower'`is a 2D matrix with empirically estimated
marginal mean power for sample sizes and target \(\Delta_\beta\)s
(averaged over simulated data sets). The second
attribute'`powerArray'`provides the full set of empirically estimated
marginal power for sample sizes and target \(\Delta_\beta\)s for each
simulated data sets in a 3D matrix. The third
attribute'`deltaArray'`contains a 3D matrix with simulated
\(\Delta_\beta\)s for sample sizes, target \(\Delta_\beta\), and
simulated data sets. The last attribute'`metric'' contains 2D matrices
with the marginal type I error rate (marTypeI), power in the classical
sense (classicalPower), actual FDR (FDR), False Discovery Cost (FDC),
and probabilities of identifying at least one true positive in table
format, where sample sizes are shown as rows and effect sizes are
columns. Examples results can be found in the example section.

\section{Runtime}\label{runtime}

In general, the computational complexity of pwrEWAS depends on four
major components: (1) assumed number and magnitude of sample size(s),
(2) number of target 's (effect sizes), (3) number of CpGs tested, and
(4) number of simulated data sets. To enhance the computational
efficiency, pwrEWAS allows users to process simulations in parallel.
While (1) and (2) are usually dictated by the study to be conducted, (3)
and (4) can be modified to either increase the precision of power
estimates (increased run time) or reduce the computational burden
(decreased precision of estimates). The following table provides the run
time of pwrEWAS for different combinations of sample sizes and effect
sizes. In all scenarios presented the number of tested CpGs was assumed
to be 100,000, number of simulated data sets was 50, and the method to
perform the differential methylation analysis as limma. A total of 6
clusters/threads were used.

\begin{longtable}[]{@{}llll@{}}
\toprule
Total sample size ~Effect size \(\Delta_\beta\) & 0.1 & 0.1, 0.2 & 0.1,
0.3, 0.5\tabularnewline
\midrule
\endhead
10 & 2min 21sec & 3min 11sec & 3min 50sec\tabularnewline
100 & 6min 22sec & 7min 39sec & 8min 33sec\tabularnewline
500 & 24min 43sec & 27min 36sec & 29min 22sec\tabularnewline
10-100 (increments of 10) & 9min 40sec & 16min 34sec & 23min
44sec\tabularnewline
300-500 (increments of 100) & 27min 58sec & 30min 01sec & 52min
00sec\tabularnewline
\bottomrule
\end{longtable}

\section{Example}\label{example}

\subsection{Running pwrEWAS}\label{running-pwrewas}

Running pwrEWAS by providing target maximal difference in methylation:

\begin{Shaded}
\begin{Highlighting}[]
\KeywordTok{library}\NormalTok{(pwrEWAS)}
\KeywordTok{set.seed}\NormalTok{(}\DecValTok{1234}\NormalTok{)}
\NormalTok{results_targetDelta =}\StringTok{ }\KeywordTok{pwrEWAS}\NormalTok{(}\DataTypeTok{minTotSampleSize =} \DecValTok{20}\NormalTok{,}
                    \DataTypeTok{maxTotSampleSize =} \DecValTok{260}\NormalTok{, }
                    \DataTypeTok{SampleSizeSteps =} \DecValTok{40}\NormalTok{,}
                    \DataTypeTok{NcntPer =} \FloatTok{0.5}\NormalTok{,}
                    \DataTypeTok{targetDelta =} \KeywordTok{c}\NormalTok{(}\FloatTok{0.02}\NormalTok{, }\FloatTok{0.10}\NormalTok{, }\FloatTok{0.15}\NormalTok{, }\FloatTok{0.20}\NormalTok{),}
                    \DataTypeTok{J =} \DecValTok{100000}\NormalTok{, }
                    \DataTypeTok{targetDmCpGs =} \DecValTok{2500}\NormalTok{, }
                    \DataTypeTok{tissueType =} \StringTok{"Blood adult"}\NormalTok{,}
                    \DataTypeTok{detectionLimit =} \FloatTok{0.01}\NormalTok{,}
                    \DataTypeTok{DMmethod =} \StringTok{"limma"}\NormalTok{,}
                    \DataTypeTok{FDRcritVal =} \FloatTok{0.05}\NormalTok{,}
                    \DataTypeTok{core =} \DecValTok{4}\NormalTok{,}
                    \DataTypeTok{sims =} \DecValTok{50}\NormalTok{)}
\end{Highlighting}
\end{Shaded}

Running pwrEWAS by providing standard deviation of difference in
methylation:

\begin{Shaded}
\begin{Highlighting}[]
\KeywordTok{library}\NormalTok{(pwrEWAS)}
\KeywordTok{set.seed}\NormalTok{(}\DecValTok{1234}\NormalTok{)}
\NormalTok{results_deltaSD =}\StringTok{ }\KeywordTok{pwrEWAS}\NormalTok{(}\DataTypeTok{minTotSampleSize =} \DecValTok{20}\NormalTok{,}
                    \DataTypeTok{maxTotSampleSize =} \DecValTok{260}\NormalTok{, }
                    \DataTypeTok{SampleSizeSteps =} \DecValTok{40}\NormalTok{,}
                    \DataTypeTok{NcntPer =} \FloatTok{0.5}\NormalTok{,}
                    \DataTypeTok{deltaSD =} \KeywordTok{c}\NormalTok{(}\FloatTok{0.00390625}\NormalTok{, }\FloatTok{0.02734375}\NormalTok{, }\FloatTok{0.0390625}\NormalTok{, }\FloatTok{0.052734375}\NormalTok{),}
                    \DataTypeTok{J =} \DecValTok{100000}\NormalTok{, }
                    \DataTypeTok{targetDmCpGs =} \DecValTok{2500}\NormalTok{, }
                    \DataTypeTok{tissueType =} \StringTok{"Blood adult"}\NormalTok{,}
                    \DataTypeTok{detectionLimit =} \FloatTok{0.01}\NormalTok{,}
                    \DataTypeTok{DMmethod =} \StringTok{"limma"}\NormalTok{,}
                    \DataTypeTok{FDRcritVal =} \FloatTok{0.05}\NormalTok{,}
                    \DataTypeTok{core =} \DecValTok{4}\NormalTok{,}
                    \DataTypeTok{sims =} \DecValTok{50}\NormalTok{)}
\end{Highlighting}
\end{Shaded}

If pwrEWAS is excecuted with providing target maximal difference, first
\(\tau\) will be determined. The beginning and finish of this process
will be printed with time stamps (see below for an example). If the
standard deviation of difference is provided, this step will be skipped.
\textbackslash{} Next, pwrEWAS will run the simulations to empirically
estimate power. pwrEWAS will indicate when the simulations are started.
To monitor the process pwrEWAS will display an process bar. pwrEWAS will
print a statement including a time stamps once finished (see below for
an example).

\begin{Shaded}
\begin{Highlighting}[]
\NormalTok{## [2019-02-12 18:40:23] Finding tau...done [2019-02-12 18:42:53]}
\NormalTok{## [1] "The following taus were chosen: 0.00390625, 0.02734375, 0.0390625, 0.052734375"}
\NormalTok{## [2019-02-12 18:42:53] Running simulation}
\NormalTok{## |===================================================================| 100%}
\NormalTok{## [2019-02-12 18:42:53] Running simulation ... done [2019-02-12 19:27:03]}
\end{Highlighting}
\end{Shaded}

\subsection{Outputs}\label{outputs}

Running pwrEWAS will result in an object, that stores the following four
attributes:

\begin{Shaded}
\begin{Highlighting}[]
\KeywordTok{attributes}\NormalTok{(results_targetDelta)}
\NormalTok{## $names}
\NormalTok{## [1] "meanPower"  "powerArray" "deltaArray" "metric"}
\end{Highlighting}
\end{Shaded}

\subsubsection{meanPower}\label{meanpower}

The primary results will be provided in the attribute `'meanPower'`. It
is essentially a summary of the attribute'`powerArray'`. meanPower will
be provide a 7x4 table with the average power by total sample size as
rows (here 20-260 patients with increments of 40) and by target
\(\Delta_\beta\), if'`targetDelta'`was provided,
or'`\(SD(\Delta_\beta)\)'', if deltaSD was provided, as columns (here
targetDelta was provided as: 0.02, 0.1, 0.15, 0.2):

\begin{Shaded}
\begin{Highlighting}[]
\KeywordTok{dim}\NormalTok{(results_targetDelta$meanPower)}
\NormalTok{## [1] 7 4}
\KeywordTok{print}\NormalTok{(results_targetDelta$meanPower)}
\NormalTok{##             0.02       0.1      0.15       0.2}
\NormalTok{## 20  0.0005415101 0.1596165 0.2758319 0.3801848}
\NormalTok{## 60  0.0863276853 0.5026172 0.6166725 0.7001472}
\NormalTok{## 100 0.1978966524 0.6402466 0.7322670 0.7947203}
\NormalTok{## 140 0.2919669218 0.7201027 0.7940429 0.8414375}
\NormalTok{## 180 0.3592038789 0.7700964 0.8317636 0.8739818}
\NormalTok{## 220 0.4201022535 0.8068096 0.8588536 0.8945975}
\NormalTok{## 260 0.4609956067 0.8338529 0.8816222 0.9117306}
\end{Highlighting}
\end{Shaded}

\subsubsection{powerArray}\label{powerarray}

The attribute `'powerArray'`should primarily be used to create a power
plot, but can also be used to investigate the power results for the
individual simulations. pwrEWAS includes a
function'`pwrEWAS\_powerPlot'`that will create a power plot, where power
(y-axis) is shown as a function of sample sizes (x-axis) for different
effect sizes (color coded). For each sample size, the mean power as well
as the 95\%tile interval (2.5\% and 97.5\%) is shown.'`sd'`should be set
to'`FALSE'`if'`targetDelta'`was specified in pwrEWAS,
and'`TRUE'`if'`deltaSD'' was specified in pwrEWAS.

\begin{Shaded}
\begin{Highlighting}[]
\KeywordTok{dim}\NormalTok{(results_targetDelta$powerArray) }\CommentTok{# simulations x sample sizes x effect sizes}
\NormalTok{## [1] 50  7  4}
\KeywordTok{pwrEWAS_powerPlot}\NormalTok{(results_targetDelta$powerArray, }\DataTypeTok{sd =} \OtherTok{FALSE}\NormalTok{)}
\end{Highlighting}
\end{Shaded}

\begin{adjustwidth}{\fltoffset}{0mm}
\includegraphics{vignette2.2_files/figure-latex/example power plot-1} \end{adjustwidth}

\subsubsection{deltaArray}\label{deltaarray}

The third attribute `'deltaArray'`contains the simulated differences in
mean DNAm. Each \(\Delta_\beta\) is drawn from a truncated normal, where
either the standard devation is provided ('`deltaSD'`) or automatically
determined based on the user-specified target \(\Delta_\beta\)
('`targetDelta'`) and the expected number of differentially methylated
CpGs ('`targetDmCpGs''). To automatically determined the standard
devation, it is adjusted stepwise until the 99.99th percentile of the
absolute value of simulated \(\Delta_\beta\)s falls within a range
around the targeted maximal difference in DNAm (see paper for additional
details). The maximal value of \(\Delta_\beta\) can exceed the
user-specified target \(\Delta_\beta\), but about 99.99\% of simulated
differences will be below user-specified target \(\Delta_\beta\) (as
seen below):

\begin{Shaded}
\begin{Highlighting}[]
\CommentTok{# maximum value of simulated differences by target value}
\KeywordTok{lapply}\NormalTok{(results_targetDelta$deltaArray, max)}
\NormalTok{## $`0.02`}
\NormalTok{## [1] 0.02095302}
\NormalTok{## }
\NormalTok{## $`0.1`}
\NormalTok{## [1] 0.1265494}
\NormalTok{## }
\NormalTok{## $`0.15`}
\NormalTok{## [1] 0.2045638}
\NormalTok{## }
\NormalTok{## $`0.2`}
\NormalTok{## [1] 0.2458416}

\CommentTok{# percentage of simulated differences to be within the target range}
\KeywordTok{mean}\NormalTok{(results_targetDelta$deltaArray[[}\DecValTok{1}\NormalTok{]] <}\StringTok{ }\FloatTok{0.02}\NormalTok{) }
\NormalTok{## [1] 0.9999999}
\KeywordTok{mean}\NormalTok{(results_targetDelta$deltaArray[[}\DecValTok{2}\NormalTok{]] <}\StringTok{ }\FloatTok{0.10}\NormalTok{)}
\NormalTok{## [1] 0.9998882}
\KeywordTok{mean}\NormalTok{(results_targetDelta$deltaArray[[}\DecValTok{3}\NormalTok{]] <}\StringTok{ }\FloatTok{0.15}\NormalTok{)}
\NormalTok{## [1] 0.9999386}
\KeywordTok{mean}\NormalTok{(results_targetDelta$deltaArray[[}\DecValTok{4}\NormalTok{]] <}\StringTok{ }\FloatTok{0.20}\NormalTok{)}
\NormalTok{## [1] 0.9999539}
\end{Highlighting}
\end{Shaded}

To get a better understanding of how the differences in mean DNAm are
distributed, pwrEWAS provides a density plot, where the distribution of
siumlated differences in mean DNAm is plotted by target differences in
DNAm (\(\Delta_\beta\)). The color theme matches the colors of the power
plot. Simulated differences within the detection limit around zero are
removed, as they are here not defined as meaningful differences.
`'sd'`should be set to'`FALSE'`if'`targetDelta'`was specified in
pwrEWAS, and'`TRUE'`if'`deltaSD'' was specified in pwrEWAS. \newpage

\begin{Shaded}
\begin{Highlighting}[]
\KeywordTok{pwrEWAS_deltaDensity}\NormalTok{(results_targetDelta$deltaArray, }\DataTypeTok{detectionLimit =} \FloatTok{0.01}\NormalTok{, }\DataTypeTok{sd =} \OtherTok{FALSE}\NormalTok{)}
\end{Highlighting}
\end{Shaded}

\begin{adjustwidth}{\fltoffset}{0mm}
\includegraphics{vignette2.2_files/figure-latex/example density plot-1} \end{adjustwidth}

In the figure above, the densities are very compress, because the first
effect size is clearly different from the other three. The following
code will provide the figure after removing the first effect size:

\begin{Shaded}
\begin{Highlighting}[]
\NormalTok{temp =}\StringTok{ }\NormalTok{results_targetDelta$deltaArray}
\NormalTok{temp[[}\DecValTok{1}\NormalTok{]] =}\StringTok{ }\OtherTok{NULL}
\KeywordTok{pwrEWAS_deltaDensity}\NormalTok{(temp, }\DataTypeTok{detectionLimit =} \FloatTok{0.01}\NormalTok{, }\DataTypeTok{sd =} \OtherTok{FALSE}\NormalTok{)}
\end{Highlighting}
\end{Shaded}

\begin{adjustwidth}{\fltoffset}{0mm}
\includegraphics{vignette2.2_files/figure-latex/example density plot w/o 0.02-1} \end{adjustwidth}

\newpage

\subsubsection{metric}\label{metric}

The fourth attribute `'metric'`contains tables on marginal type I error
rate ('`marTypeI''), power in the classical sense (classicalPower),
actual FDR (FDR), False Discovery Cost (FDC, see paper for additional
details), and probabilities of identifying at least one true positive,
for each sample size and effect size combination:

\begin{Shaded}
\begin{Highlighting}[]
\NormalTok{results_targetDelta$metric}
\NormalTok{## $marTypeI}
\NormalTok{##             0.02          0.1         0.15          0.2}
\NormalTok{## 20  0.0000000000 0.0001927742 0.0003533407 0.0004575820}
\NormalTok{## 60  0.0003435644 0.0006813394 0.0008155174 0.0009199059}
\NormalTok{## 100 0.0011254329 0.0008494543 0.0009544978 0.0010784126}
\NormalTok{## 140 0.0023155015 0.0009987010 0.0011007120 0.0011301504}
\NormalTok{## 180 0.0031646165 0.0010936404 0.0011537869 0.0011709668}
\NormalTok{## 220 0.0043017549 0.0011711588 0.0011688609 0.0011961111}
\NormalTok{## 260 0.0050251766 0.0011825572 0.0012256528 0.0012703139}
\NormalTok{## }
\NormalTok{## $classicalPower}
\NormalTok{##          0.02       0.1      0.15       0.2}
\NormalTok{## 20  0.0000230 0.1140913 0.2188748 0.3211014}
\NormalTok{## 60  0.0072840 0.3665948 0.4969722 0.5948422}
\NormalTok{## 100 0.0243978 0.4749589 0.5952528 0.6816387}
\NormalTok{## 140 0.0447472 0.5386504 0.6500822 0.7252384}
\NormalTok{## 180 0.0650878 0.5859314 0.6852212 0.7575621}
\NormalTok{## 220 0.0848528 0.6181004 0.7131985 0.7789187}
\NormalTok{## 260 0.1031464 0.6445875 0.7373262 0.7980121}
\NormalTok{## }
\NormalTok{## $FDR}
\NormalTok{##            0.02        0.1       0.15        0.2}
\NormalTok{## 20  0.000000000 0.04402833 0.04704810 0.04431709}
\NormalTok{## 60  0.004517889 0.04837274 0.04781140 0.04793149}
\NormalTok{## 100 0.004418029 0.04660417 0.04675174 0.04899762}
\NormalTok{## 140 0.004953809 0.04824695 0.04925629 0.04831297}
\NormalTok{## 180 0.004662894 0.04856488 0.04900589 0.04792516}
\NormalTok{## 220 0.004844068 0.04925471 0.04774217 0.04763118}
\NormalTok{## 260 0.004668246 0.04777146 0.04839480 0.04928052}
\NormalTok{## }
\NormalTok{## $FDC}
\NormalTok{##           0.02        0.1       0.15        0.2}
\NormalTok{## 20  0.00000000 0.04638179 0.04959881 0.04652390}
\NormalTok{## 60  0.03043364 0.05207772 0.05098151 0.05083022}
\NormalTok{## 100 0.04361648 0.05082257 0.05032126 0.05243835}
\NormalTok{## 140 0.06076271 0.05345012 0.05358616 0.05197136}
\NormalTok{## 180 0.06718904 0.05446132 0.05373615 0.05186793}
\NormalTok{## 220 0.07820442 0.05588333 0.05261712 0.05173305}
\NormalTok{## 260 0.08348369 0.05445357 0.05369060 0.05387600}
\NormalTok{## }
\NormalTok{## $probTP}
\NormalTok{##     0.02 0.1 0.15 0.2}
\NormalTok{## 20   0.4   1    1   1}
\NormalTok{## 60   1.0   1    1   1}
\NormalTok{## 100  1.0   1    1   1}
\NormalTok{## 140  1.0   1    1   1}
\NormalTok{## 180  1.0   1    1   1}
\NormalTok{## 220  1.0   1    1   1}
\NormalTok{## 260  1.0   1    1   1}
\end{Highlighting}
\end{Shaded}

\subsection{Interpretation}\label{interpretation}

To detect differences up to 10\%, 15\% and 20\% in CpG-specific
methylation across 2,500 CpGs with at least 80\% power, we would need
about 220, 180 and 140 total subjects, respectively. As expected, 80\%
power was not achieved for a difference in DNAm up to 2\% for the
selected total sample size range. However, it can be observed that the
probability of detecting at least one CpG out of the 2500 differentially
methylated CpGs is about 40\% for 20 total patients and virtually 100\%
for 60 and more total patients.

\section{SessionInfo}\label{sessioninfo}

\begin{Shaded}
\begin{Highlighting}[]
\KeywordTok{toLatex}\NormalTok{(}\KeywordTok{sessionInfo}\NormalTok{())}
\end{Highlighting}
\end{Shaded}

\begin{itemize}\raggedright
  \item R version 3.5.2 (2018-12-20), \verb|x86_64-w64-mingw32|
  \item Locale: \verb|LC_COLLATE=English_United States.1252|, \verb|LC_CTYPE=English_United States.1252|, \verb|LC_MONETARY=English_United States.1252|, \verb|LC_NUMERIC=C|, \verb|LC_TIME=English_United States.1252|
  \item Running under: \verb|Windows 7 x64 (build 7601) Service Pack 1|
  \item Matrix products: default
  \item Base packages: base, datasets, graphics, grDevices, methods,
    stats, utils
  \item Other packages: BiocStyle~2.10.0, foreach~1.4.4,
    pwrEWAS~0.99.0, shinyBS~0.61
  \item Loaded via a namespace (and not attached):
    BiocManager~1.30.4, bookdown~0.9, codetools~0.2-16,
    colorspace~1.4-0, compiler~3.5.2, crayon~1.3.4, digest~0.6.18,
    evaluate~0.12, ggplot2~3.1.0, grid~3.5.2, gtable~0.2.0,
    htmltools~0.3.6, httpuv~1.4.5.1, iterators~1.0.10, knitr~1.21,
    later~0.7.5, lazyeval~0.2.1, magrittr~1.5, mime~0.6,
    munsell~0.5.0, pillar~1.3.1, pkgconfig~2.0.2, plyr~1.8.4,
    promises~1.0.1, R6~2.3.0, Rcpp~1.0.0, rlang~0.3.1,
    rmarkdown~1.11, rstudioapi~0.9.0, scales~1.0.0, shiny~1.2.0,
    stringi~1.2.4, stringr~1.4.0, tibble~2.0.1, tools~3.5.2,
    xfun~0.4, xtable~1.8-3, yaml~2.2.0
\end{itemize}

\section{References}\label{references}

\end{document}
